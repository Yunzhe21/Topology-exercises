\documentclass[12pt]{article}
\usepackage[margin=1in]{geometry}
\usepackage[all]{xy}


\usepackage{amsmath,amsthm,amssymb,color,latexsym}
\usepackage{geometry}        
\geometry{letterpaper}    
\usepackage{graphicx}

\newtheorem{problem}{Problem}

\newenvironment{solution}[1][\it{Solution}]{\textbf{#1. } }{$\square$}


\begin{document}
\noindent Topology \hfill Assignment 2\\
Yunzhe Zheng. (2025/02/21)

\hrulefill

\begin{problem}
When viewed as metric subspaces of $\mathbb{R}$, are $\{1, \frac{1}{2}, \frac{1}{3}, \dots\}$ and $\{1, 2, 3, \dots\}$ homeomorphic? Isometric?
\end{problem}

\textbf{Proof:} Consider the map $f: H := \{1, \frac{1}{2}, \frac{1}{3}, \dots\}\to G := \{1, 2, 3, \dots\}$ by $f(x) = \frac{1}{x}$, and obviously it is a bijection. Notice that any subset in $H$ (or $G$) is open in $H$ (or $G$), since for arbitrary $x\in H$, Consider the ball $B_{1/2}(x)$, then $B_{1/2}(x)\cap H=\{x\}\subseteq H$. For $O\subseteq G$, which is automatically open, $f^{-1}(O)$ is open, and vice versa. Thus is homeomorphic. \\
\indent For isometry, notice that for $x,y\in H$, $d_{H}(x,y)=|x-y| < 1$. However, $2-1=1$, then they are not isometric. \qed \\

\begin{problem}
Given a metric space $X$, and any point $x$ and $y$ in $X$, define 
$$
\Lambda_{X}(x,y):= \{z\in X: d_{X}(x,y)=d_{X}(x, z) + d_{X}(z,y)\}.
$$
\indent a). Show that if $f$ is an isometry from $X$ onto a metric space $Y$, then 
$$
\Lambda_{Y}(f(x), f(y))=f(\Lambda_{X}(x,y)) \text{, for all } x,y\in X.
$$ 
\indent b). Show that $\mathbb{R}_{1}^{n}$ and $\mathbb{R}^{n}_{2}$ are not isometric for any $n\geq 2$
\end{problem}

\textbf{Proof:} a). For $z\in \Lambda_{x}(x,y)$, then $d_{X}(x,y) = d_{X}(x,z) + d_{X}(z,y)$, then by isometry of $f$, $d_{Y}(f(x), f(y))=d_{Y}(f(x), f(z))+d_{Y}(f(z), f(y))$, which means that $f(z)\in\Lambda_{Y}(f(x), f(y))$, implying $f(\Lambda_{x}(x,y))\subseteq \Lambda_{Y}(f(x), f(y))$. Conversely, for $z'\in \Lambda_{Y}(f(x), f(y))$, $d_{Y}(f(x), f(y))=d_{Y}(f(x), z')+d_{Y}(z', f(y))$. By injectivity condition, there exists unique $z\in X$ such that $f(z)=z'$, and $d_{X}(x, y)=d_{X}(x, z)+d_{X}(z, y)$ by isometry, then $z\in \Lambda_{X}(x, y)$, which implies $z'\in f(\Lambda_{X}(x,y))$, hence we have $\Lambda_{Y}(f(x), f(y))=f(\Lambda_{x}(x,y))$ for every $x,y\in X$. \\
\indent b). If $f$ is an isometry from $\mathbb{R}_{1}^{n}$ onto $\mathbb{R}^{n}_{2}$, then $f-f(\mathbf{0})$ is also a isometry, then we may assume $f(\mathbf{0})=\mathbf{0}$ without loss of generality, since we can always do translation as above. Consider $C_{1} = \{x\in \mathbb{R}^{n}_{1}: \|x\|_{1}=1\}$ and $C_{2}=\{x\in\mathbb{R}^{n}_{1}: \|x\|_{2}=1\}$, then $f$ send $C_{1}$ onto $C_{2}$ by isometry to the origin. Consider vertices of $C_{1}$, $v_{i}=(0, 0, \dots, 0, 1, 0, \dots, 0)$, where only the i-th component is $1$, then notice that every two distinct vertices in $C_{1}$ has distance $2$. Now without loss of generality, assume that $f$ maps $(1, 0, \dots, 0)$ to $(1, 0, \dots, 0)=:x$, then for $y=(y_{1}, y_{2}, \dots, y_{n})\in C_{2}$, $d_{2}(x,y)=\sqrt{(y_{1}-1)^{2}+y_{2}^{2}+\dots+y_{n}^{2}}=2$, combining with the fact that $y_{1}^{2}+\dots+y_{n}^{2}=1$, obtain that $y_{1}=-1$ and $y_{i}=0$ otherwise. However, we have more than one candidate for that point, which breaks the injectivity of $f$. Thus $f$ cannot be an isometry. \qed
\\
\begin{problem}
Let $A$ and $B$ be two finite subsets of $\mathbb{R}^{n}$ with $|A|=|B|$. Prove that $\mathbb{R}^{n}\setminus A\simeq\mathbb{R}^{n}\setminus B$.
\end{problem}

\textbf{Proof:} Denote $A=\{a_{1}, \dots, a_{n}\}$, and $B=\{b_{1}, \dots, b_{n}\}$. It is sufficient to show that there exists a homeomorphism between $\mathbb{R}^{n}$ and $\mathbb{R}^{n}$ such that it sends $A$ to $B$. In order to utilize induction, we should prove that for $A'\subset\mathbb{R}^{n}$, which is a set of finite points, and $x,y\notin A'$, there exists a homeomorphism $f$ such that $f(x)=y$. Since $A'$ is finite, there exists $z\in \mathbb{R}^{n}$ such that the union of two line segments $\overline{xz}\cup\overline{yz}$ contains no point in $A'$, then there exits $\epsilon$ such that $U:=\{p\in \mathbb{R}^{n}: d(p, \overline{xz}\cup\overline{yz})<\epsilon\}$ and $U\cap A'=\emptyset$. Since $\overline{U}$ is homeomorphic to a closed ball, then we will prove later that there exists a homeomorphism $\varphi$ on $\overline{U}$ such that $\varphi(x)=y$ and $\varphi(u)=u$ when $x\in \partial U$. Now, induct on the cardinality of $A$. When $n=1$, apply translation. Suppose for $n\leq N$, we have the conclusion, then for $n=N+1$, first use induction step, there exists homeomorphism $f(a_{i})=b_{i}$ for $1\leq i\leq N$, also, by previous claim, we have homeomorphism $\varphi(a_{N+1})=b_{N+1}$ and $\varphi$ is identity outside of the interior of some neighborhood not intersecting any previous points $a_{i}$, $1\leq i\leq N$. Hence $f\circ \varphi$ is the desired homeomorphism. \\
\indent We are left to show that such $\varphi$ does exist. We first consider the map from closed unit ball to closed unit ball. Given $t\alpha\in \overline{B}$, $\|\alpha\|=1$ and $t\in [0, 1]$, define $\varphi(t\alpha)=t\alpha+(1-t)p$, then it satisfies our criterion. $\overline{B}$ after homeomorphism generalize this $\varphi$ to the $U$ mentioned before in the proof, and thanks to problem 6, on the boundary of $U$, the map is still identity. \qed
\\
\begin{problem}
Let $n$ be a positive integer. Show that $\{x\in\mathbb{R}^{n}: \|x\|_{p}=1\}\simeq \mathbb{S}^{n-1}$ for any $p\in[1, \infty]$
\end{problem}

\textbf{Proof:} Consider the map $f: \{x\in\mathbb{R}^{n}: \|x\|_{p}=1\} \to \mathbb{S}^{n-1}$ as $f(x)=\frac{x}{\|x\|_{2}}\in \mathbb{S}^{n-1}$, and claim its inverse is 
 $g: \mathbb{S}^{n-1}\to \{x\in\mathbb{R}^{n}: \|x\|_{p}=1\}$ as $g(y)=\frac{y}{\|y\|_{p}}$. Indeed, $f(g(y))=\frac{y/\|y\|_{P}}{\|y/\|y\|_{p}\|_{2}}=\frac{y}{\|y\|_{2}}=y$, and similarly with $g(f(x))$. Also, map $f,g$ are continuous given that the norm is a continuous function, thus the two spaces are homeomorphic. \qed
\\
\begin{problem}
Show that the products of countably many homeomorphic metric spaces is homeomorphic. Conclude that if we metrized $[0,1]\times [0,\frac{1}{2}]\times \dots$ by the product metric, we would obtain a metric space that is homeomorphic to the Hilbert cube. Now combine this fact with Theorem 3.1 to conclude: Every separable metric space can be embedded in $l^{2}$.
\end{problem}

\textbf{Proof:} Suppose we have $A_{i}\simeq B_{i}$ for $i=1, 2, 3\dots$, and denote each homeomorphism by $f_{i}: A_{i}\to B_{i}$. Consider $F: \prod\limits_{i=1}^{\infty}A_{i}\to \prod\limits_{i=1}^{\infty}B_{i}$ by $F(a_{1}, a_{2, \dots})=(f_{1}(a_{1}), f_{2}(a_{2}),\dots)$, and this map is clearly bijective since each component function is a homeomorphism. Continuity of $F$ is given by choosing any $(x_{m})_{m=1}^{\infty}\in \prod\limits_{i=1}^{\infty}A_{i}$ that converges to $x$, then $x_{m,i}\to x_{i}$ for all $i$, hence $F(x_{m})\to (f_{1}(x_{1}), f_{2}(x_{2}), \dots)=F(x)$ continuous, the proof for continuity of $F^{-1}$ is identical once we realize each $f^{-1}_{i}$ is continuous by homeomorphism condition. \\
\indent For $[0, \frac{1}{2^n}]$ for $n\geq 0$, consider $f(x)=2^{n}x$, then it's obvious that $[0, \frac{1}{2^n}]$ and $[0, 1]$ is homeomorphic, then by previous result $[0, 1]\times [0,\frac{1}{2}]\times\dots\simeq[0, 1]^{\infty}$. Finally, it's sufficient to prove that $[0, 1]\times [0, \frac{1}{2}]\times\dots$ is embedded in $l^{2}$, and since it is also homeomorphic to $H=\prod\limits_{i=1}^{\infty}[0, \frac{1}{2^{i}}]$, consider the identity map $I: H\to l^{2}$ (which obviously is well-defined). Surely it is injective, and for continuity, fix $\epsilon >0$, take $\delta<\epsilon^2$, whenever $\rho(x,y)<\delta$, $d_{2}(x,y)=\left( \sum\limits_{i=1}^{\infty}|x_{i}-y_{i}|^2\right)^{1/2}\leq\left(\sum\limits_{i=1}^{\infty}2^{-i}|x_{i}-y_{i}|\right)^{1/2}=(\rho(x,y))^{1/2}<\epsilon$, then $I$ is continuous. On the other hand, consider $(x_{m})_{m=1}^{\infty}\in l^{2}$ such that $x_{m}\to x$, then there exists a $M>0$ such that whenever $m > M$, we have $\left(\sum\limits_{i=1}^{\infty}|x_{m,i}-x_{i}|^{2}\right)^{1/2}<\epsilon$, which implies that $|I^{-1}(x_{m,i})-I^{-1}(x_{i})|=|x_{m, i}-x_{i}|<\epsilon$ for all $i$, then $(I^{-1}(x_{m}))_{m=1}^{\infty}$ converges component-wisely to $I^{-1}(x)$, $I^{-1}(x_{m})\to I^{-1}(x)$, $I^{-1}$ is also continuous. To conclude, we find $[0, 1]\times [0, \frac{1}{2}]\times\dots\simeq H\simeq l^{2}$, hence by Theorem 3.1 which states that every separable metric space is embedded in $[0,1]^{\infty}$, they are also embedded in $l^{2}$. \qed
\\
\begin{problem}
Let $X$ and $Y$ be two metric spaces, and $f: X\to Y$ a homeomorphism. Prove that $f(\text{cl}(S))=\text{cl}(f(S))$ and $f(\partial S)=\partial(f(S))$ for any subset $S$ of $X$.
\end{problem}

\textbf{Proof:} For $y\in f(\overline{S})$, then $y = f(x)$ for $x\in \text{int}(S)$ or $x\in \partial S$. In the former case, $y\in \overline{f(S)}$, otherwise, for every $\epsilon >0$, $B(x, \epsilon)\cap S\neq\emptyset$. By setting $\epsilon_{n} = \frac{1}{n}$, we get a sequence $x_{m}\to x$, then by continuity of $f$, $f(x_{m})\to f(x)$. We claim that $f(x)\in \overline{f(S)}$, because otherwise there exists a ball centered at $f(x)$ that contained completely in $Y\setminus \overline{f(S)}$, contradicting $f(x_{m})\to f(x)$, hence $f(\overline{S})\subseteq \overline{f(S)}$. Conversely, suppose $y\in \overline{f(S)}$, the $y\in \text{int}(f(S))$ or $y\in \partial(f(S))$, the former case implies $y\in f(\overline{S})$. When $y\in \partial(f(S))$, for all $\epsilon >0$, there exists $B(y, \epsilon)$ such that $B(y, \epsilon)\cap f(S)\neq\emptyset$, then we obtain $y_{m}\to y$, and by homeomorphism condition, $f^{-1}(y_{m})\to f^{-1}(y)$. $f^{-1}(y)$ must be in $\overline{S}$, for otherwise, there exists a ball containing $f^{-1}(y)$ that is contained in $X\setminus \overline{S}$, contradicting convergence condition. Thus $f(\overline{S})=\overline{f(S)}$. \\
\indent For the latter part, since $f$ is bijective, $f(\partial(S))=f(\overline{S}\setminus\text{int}(S))=f(\overline{S})\setminus f(\text{int}(S))=\overline{f(S)}\setminus f(\text{int}(S))$. We are left to show that $\text{int}(f(S))=f(\text{int(S)})$, to see that, notice $f(\text{int}(S))=f(X\setminus \overline{S^{c}})$, where $S^{c}$ is the complement of $S$ in $X$, then $f(\text{int}(S))=f(X)\setminus f(\overline{S^{c}})=f(X)\setminus \overline{f(S^{c})}=\text{int}(f(S))$. Hence $f(\partial(S))=f(\overline{S}\setminus\text{int}(S))=f(\overline{S})\setminus f(\text{int}(S))=\partial(f(S))$. \qed


\end{document}