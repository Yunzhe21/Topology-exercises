\documentclass[12pt]{article}
\usepackage[margin=1in]{geometry}
\usepackage[all]{xy}


\usepackage{amsmath,amsthm,amssymb,color,latexsym}
\usepackage{geometry}        
\geometry{letterpaper}    
\usepackage{graphicx}

\newtheorem{problem}{Problem}

\newenvironment{solution}[1][\it{Solution}]{\textbf{#1. } }{$\square$}


\begin{document}
\noindent Topology \hfill Assignment 5\\
Yunzhe Zheng. (2025/04/18)

\hrulefill

\begin{problem}
Let $\sim$ be the equivalent relation on $\mathbb{D}^{2}$ defined by $x\sim y$ iff $x=\mp y$. Show that $\mathbb{D}^2/_\sim\simeq \mathbb{D}^2$.
\end{problem}

\textbf{Proof:} Consider the map $f: \mathbb{D}^2\to\mathbb{D}^2$ as $re^{i\theta}\mapsto r^2e^{2i\theta}$, then $x\sim y$ if and only if $f(x)=f(y)$. We only need to show that it is a open continuous surjective map. Continuity is clear from construction. For surjectivity, notice that when $z=re^{i\theta}\in\mathbb{D}^2$, $f(\sqrt{r}e^{i\theta/2})=z$, where $\sqrt{r}e^{i\theta/2}\in \mathbb{D}^2$. For openness, Consider any open set $U$ containing a point $z\in\mathbb{D}^2$, and $z'\in U$, if $z'=0$, then for any ball $B_{\epsilon}(z)\subseteq U$,$f(B_\epsilon(z))=B_{\epsilon^2}(z)$, which is open. Otherwise, since the map sends one point and its opposite point to the same one, then for some $O=B_{\delta}(z')\cap U$, map $f|_{B_{\delta}(z')}$ has an inverse function and is continuous, $f|_{B_{\delta}(z')}(O)$ is open, thus $f$ is a open function. \qed
\\
\begin{problem}
Let $X$ be a nonempty set. Let $d: X\times X\to [0,+\infty)$ be a function that satisifies the Symmetry and Triangle inequality. Assume, in addition, that $d(x,x)=0$ for each $x\in X$. \\
\indent a. Show that the binary relation $\sim$ on $X$, defined by $x\sim y$ iff $d(x,y)=0$ is an equivalent relation. \\
\indent b. Define the map $D: X\times X\to \mathbb{R}_{+}$ by $D([x]_{\sim}, [y]_{\sim}):= d(x,y)$. Show that this is a well-defined metric on $X/_\sim$. \\
\indent c. The topology induced by $d$ is defined analogously to how the metric topology is defined. (That is, we define the open balls, and hence open sets, in $X$ relative to the semimetric.) Regarding $X$ as a topological space relative to this ``semimetric topology", prove that the quotient topology on $X/_\sim$ coincides with the metric topology on $X/_\sim$ induced by $D$.
\end{problem}

\textbf{Proof:} a. Firstly since $d(x,x)=0$ for each $x\in X$, have $x\sim x$. Secondly, if $d(x,y)=0$, then $d(y,x)=0$, implying that if $x\sim y$ then $y\sim x$. Finally if $x\sim y$, $y\sim z$, then $d(x,y)=0$ and $d(y,z)=0$, by triangle inequality, $0\leq d(x,z)\leq d(x,y)+d(y,z)=0$, which implies that $x\sim z$, thus is indeed a equivalent relation. \\
\indent b. Whenever $[x]_\sim=[x']_\sim$ and $[y]_{\sim}=[y']_\sim$, $D([x']_\sim,[y']_\sim)=d(x',y')=d(x',x)+d(x,y)+d(y,y')=d(x,y)=D([x]_\sim,[y]_\sim)$, thus is well-defined. To show it is a metric, firstly, $D([x]_\sim, [y]_\sim)=d(x,y)\geq 0$, and if $D([x]_\sim,[y]_\sim)=0$, then $d(x,y)=0$, which implies that $[x]_\sim=[y]_\sim$. Conversely when $[x]_\sim=[y]_\sim$, $D([x]_\sim, [y]_\sim)=d(x,y)=0$ since they are in the same equivalent class. Secondly, $D([x]_\sim,[y]_\sim)=d(x,y)=d(y,x)=D([y]_\sim, [x]_\sim)$. Finally, $D([x]_\sim, [z]_\sim)=d(x,z)\leq d(x,y)+d(y,z)=D([x]_\sim, [y]_\sim)+D([y]_\sim, [z]_\sim)$ \\
\indent c. For any open set $O$ from the quotient topology on $X/_\sim$, $\bigcup O$ is open in $X$, then for any $[x]_\sim\in O$, there exists $B_{\epsilon}(x)\subseteq\bigcup O$, where $B_\epsilon(x)=\{y\in\bigcup O: D([y]_\sim, [x]_\sim)<\epsilon\}=\bigcup\{[y]_\sim\in O: D([y]_\sim, [x]_\sim)<\epsilon\}$, then $[x]_\sim\in\{[y]_\sim\in O: D([y]_\sim, [x]_\sim)<\epsilon\}\subseteq O$, $O$ can be written as the union of open balls in the metric topology on $X/_\sim$ induced by $D$. Conversely, for any $\{[y]_\sim: D([y]_\sim,[x]_\sim)<\epsilon\}$, $\bigcup\{[y]_\sim: D([y]_\sim,[x]_\sim)<\epsilon\}=B_\epsilon(x)$ is open in $X$, thus it is open in quotient topology sense. \qed
\\
\begin{problem}
Put $S=\{1, \frac{1}{2}, \frac{1}{3}, \dots\}$ and let $X$ be the set of all real numbers endowed with the following topology: \\
$$
\{O\setminus T: O\in\mathcal{O}_{\mathbb{R}}\text{ and } T\subseteq S\}.
$$\\
\indent a. Check that $X$ is Hausdorff, and $S$ is a closed subset of $X$. \\
\indent b. Consider the equivalent relation $\sim$ on $X$ defined by $x\sim y$ iff either $x=y$ or $\{x,y\}\subseteq S$. Show that $S/_\sim$ is a $T_1$-space. \\
\indent c. Show that $X/_\sim$ is not Hausdorff, even though $\sim$ is a closed subset of $X\times X$.
\end{problem}

\textbf{Proof:} a. For arbitrary $x<y\in X$, consider two open set $O=(-\infty,\frac{x+y}{2})\setminus (S\setminus\{x\})$ and $U=(\frac{x+y}{2}, +\infty)\setminus (S\setminus\{y\})$, then $x\in O$ and $y\in U$, where $O,U$ are disjoint and both open in the topology defined, thus $X$ is Hausdorff. $S^c=X\setminus S$, which is obviously open, thus $S$ is closed subset of $X$. \\
\indent b. It's sufficient to show that $\{[x]_\sim\}$ is closed, which is equivalent to say that $\bigcup [x]_\sim$ is closed. When $x\in S$, then $[x]_\sim=S$, then $\bigcup[x]_\sim$ is closed. Otherwise, $[x]_\sim = \{x\}$, and $\{x\}$ is closed because $(-\infty, x)\cup(x,+\infty)$ is open, hence $S/_\sim$ is $T_1$-space. \\
\indent c. Consider $[0]_\sim=\{0\}$ and $[1]_\sim=S$. If there exists disjoint $U,V\in \mathcal{O}_{X_\sim}$ such that $[0]_\sim\in U$ and $[1]_\sim\in V$, then $\bigcup V\in\mathcal{O}_{X}$ and must contain $S$. However, for any $U$, $\bigcup U$ is an open set containing $0$, which also contains $1/n$ for some $n$, hence $X/_\sim$ is not Hausdorff. \qed
\\
\begin{problem}
Show that any infinite set is connected relative to the cofinite topology.
\end{problem}

\textbf{Proof:} Suppose that the infinite set $X$ is not connected relative to cofinite topology, then there exists open set $U,V$ such that $U\cap V=\emptyset$ and $U\cup V=X$. By definition of cofinite topology, $U^c=V$ is finite, then $V$ cannot be open since the ground space is infinite, contradicting to our assumption, hence $X$ is connected under cofinite topology. \qed
\\
\begin{problem}
Prove: \\
\indent a. No subspace of $\mathbb{R}$ is homeomorphic to $\mathbb{S}^1$. \\
\indent b. No open subspace of $\mathbb{R}$ is homeomorphic to $\mathbb{R}^n$ for any $n\geq 2$. \\
\indent c. No $1$-manifold is homeomorphic to an $n$-manifold for any $n\geq 2$.
\end{problem}

\textbf{Proof:} a. If there exists a subspace $\mathcal{U}$ of $\mathbb{R}$ that is homeomorphic to $\mathbb{S}^1$ under $f$ ($f: \mathbb{S}^1\to \mathcal{U}$), then since $\mathbb{S}^1$ is connected, $\mathcal{U}$ must be connected, and thus is an interval. Pick $y\in\mathbb{S}^1$ such that $f(y)$ is not one of the endpoints, then $\mathcal{U}\setminus \{f(y)\}$ is not connected, while $\mathbb{S}^1\setminus\{y\}$ is still connected, which is absurd. \\
\indent b. Suppose there exists a subspace $\mathcal{U}$ of $\mathbb{R}$ that is homeomorphic to $R$, then $\mathcal{U}$ must be connected, specifically an open intervel. Let $f: \mathbb{R}^n\to \mathcal{U}$ be such a homeomorphism, and consider $f(0)\in\mathcal{U}$. $\mathcal{U}\setminus\{f(0)\}$ is not connected, however, $\mathbb{R}^n\setminus\{0\}$ is connected, hence it cannot be homeomorphic to $\mathbb{R}^n$. \\
\indent c. Suppose $1$-manifold $X$ is homeomorphic to $n$-manifold $Y$ for $n\geq 2$, under $f: X\to Y$, then by definition, for any $x\in X$, there exists an open neighborhood $O_x$ that is homeomorphic to $\mathbb{R}$. For $f(O_x)$, which is an open set in $Y$ containing $f(x)$, it is homeomorphic to $\mathbb{R}^n$, implying that $O_x$ is homeomorphic to $\mathbb{R}^n$, which from previous conclusion is absurd. \qed

\begin{problem}
Show that the following sets are connected in $\mathbb{R}^2$: \\
\indent a. (The comb Space) $([0,1]\times\{0\})\cup(\bigcup_{i=1}^{\infty}(\{\frac{1}{i}\}\times[0,1]))\cup(\{0\}\times[0,1])$. \\
\indent b. (Topologist's Sine Curve) $(\{0\}\times [-1, 1])\cup \{(t, \sin\frac{1}{t}): 0<t\leq 1\}$. \\
\indent c. $\{x\in\mathbb{R}^2: \text{either } x_1 \text{ or } x_2 \text{ is rational}\}$.
\end{problem}

\textbf{Proof:} a. Let $\mathcal{Y}=\{\{\frac{1}{i}\times [0,1]\}: i=1, 2, \dots\}\cup\{\{0\}\times[0,1]\}$ be the collection of connected subspaces of $\mathbb{R}^2$, and notice that $([0,1]\times \{0\})\cap S\neq\emptyset$, then the comb space is connected. \\
\indent b. Denote the whole space as $S$, then it's sufficient to prove that $C:=\overline{\{(t, \sin\frac{1}{t}): 0<t\leq 1\}}=S$, then by the fact that $\{(t, \sin\frac{1}{t}): 0<t\leq 1\}$ is path-connected (connected), $S$ is connected. To prove so, suppose for any $p=(x,y)\in S$, if $p\in \{(t, \sin\frac{1}{t}): 0<t\leq 1\}$, then there is a trivial sequence converging to it. Otherwise, $p=(0,y)$ for some $0\leq y\leq 1$. Consider sequence $(\frac{1}{\theta+2k\pi}, \sin(\theta+2k\pi))$, where $\sin(\theta)=y$, then it converges to $p$. Conversely, we have to prove that $S$ is indeed closed. Consider sequence $\{(x_n,y_n)\}\in S^\infty\to (x,y)$, then $\lim\limits_{n\to\infty}x_n=x$ and $\lim\limits_{n\to\infty}y_n=y$. If $x=0$, then by the fact that $\lim\limits_{n\to\infty}y_n\in [-1, 1]$, $(x,y)\in S$, otherwise, up to deleting points where $x_n=0$,  $y_n=\sin(\frac{1}{x_n})$, then $\lim\limits_{n\to\infty}y_n=\lim\limits_{n\to\infty}\sin(\frac{1}{x_n})=\sin(\frac{1}{x})$, hence $S$ is closed. \\
\indent c. Let $X$ be the whole space, and consider the following two subsets $S_1:=\{x\in\mathbb{R}^2: x_1\text{ rational}\}\cup \{y=0\}$ and $S_2:=\{x\in\mathbb{R}^2: x_2\text{ rational}\}\cup\{x=0\}$. For $S_1$, let $\mathcal{Y}:=\{\{x\in\mathbb{R}^2:x_{1}\text{ rational}\}\}$, and $\{y=0\}$ be its connector, then $S_1$ is connected. $S_2$ is connected for the exact same reasoning. Since $S_{1}\cup S_2=X$ and $S_1\cap S_2\neq\emptyset$, $X$ is connected in $\mathbb{R}^2$. \qed 
\\
\begin{problem}
Show that $\mathbb{R}^\infty$ is not connected relative to the box topology. Conclude that the countable product of connected topological spaces need not be connected relative to the box topology. (Hint: Check that $l^\infty$ is clopen in $\mathbb{R}^\infty$ relative to the box topology.)
\end{problem}

\textbf{Proof:} For any $a=(a_1, a_2, \dots)\in l^\infty$, we have $\sup_n |a_n|<\infty$ by definition, then there exists $O_a=(B_{1}(a_1), B_2(a_2), \dots)\in\mathcal{O}_a$, and since $\sup_n |b_n|<\infty$ for any $b\in O_a$, thus $l^\infty$ is open. Also, consider any $y\in(l^{\infty})^{c}$, $\sup_n |y_n|=+\infty$, there exists $U_y=(B_1(y_1), B_1(y_2), \dots)\in\mathcal{O}_y$ such that $\sup_n |y'_n|=\infty$ for any $y'\in U_y$, then $(l^\infty)^c$ is open, $l^\infty$ is clopen. To conclude, $\mathbb{R}^\infty$ is not connected relative to the box topology. \qed

\begin{problem}
Consider th letters of the Englosh alphabet as subspaces of $\mathbb{R}^2$. Identify which of these subspaces are homeomorphic to each other. 
\end{problem}

\textbf{Solution:} We use the fact that the number of $n$-vertices is a topological invariance. Below are $9$ equivalence classes with respect to the equivalence relation "is homeomorphic to": \\
\indent 1. C, L, M, N, S, U, V, W, Z \\
\indent 2. D, O\\
\indent 3. Q \\
\indent 4. A, R \\
\indent 5. B \\
\indent 6. E, F, G (depending on how to write this letter), J, T, Y\\
\indent 7. H, I \\
\indent 8. K, X\\
\indent 9. Q
\\
\begin{problem}
A topological space $X$ is said to be weakly locally connected if for every $x\in X$ and $O\in\mathcal{O}_X(x)$, there is a connected subset $S$ of $O$ that contains $x$ in its interior. Prove that $X$ is weakly locally connected iff it is locally connected.
\end{problem}

\textbf{Proof:} If $X$ is locally connected, then by definition, it's obvious that $X$ is weakly locally connected since every point in an open set is in its interior. Conversely suppose that $X$ is weakly locally connected, then for every $x\in X$ and $O\in\mathcal{O}_X(x)$, there is a connected subset $S$ of $O$ such that $x$ is in its interior. If $\text{int}(S)$ is connected, then there is nothing to prove, otherwise, it can be divided into connected components, and by Proposition 2.9 from textbook, each connected component is open, and one of them contains $x$, thus is locally connected. \qed
\end{document}
