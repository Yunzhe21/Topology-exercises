\documentclass[12pt]{article}
\usepackage[margin=1in]{geometry}
\usepackage[all]{xy}


\usepackage{amsmath,amsthm,amssymb,color,latexsym}
\usepackage{geometry}        
\geometry{letterpaper}    
\usepackage{graphicx}

\newtheorem{problem}{Problem}

\newenvironment{solution}[1][\it{Solution}]{\textbf{#1. } }{$\square$}


\begin{document}
\noindent Topology \hfill Assignment 4\\
Yunzhe Zheng. (2025/03/18)

\hrulefill

\begin{problem}
Let $X:= \{(x_{1}, x_{2})\in\mathbb{R}^{2}: x_{2}\geq 0\}$. For any $x:=(x_{1}, 0)\in X$, let $\mathcal{B}(x)$ be the collection of all sets that are obtained by adding $x$ to the usual open balls that are contained in $X$ and are tangent to the horizontal axis at $x$. (That is, any member of $\mathcal{B}(x)$ looks like $\{x\}\cup B((x_{1},\epsilon),\epsilon)$ where $\epsilon>0$). For any $x:=(x_{1},x_{2})\in X$ with $x_{2}>0$, put $\mathcal{B}(x):=\{B(x,\epsilon): \epsilon\in(0,|x_{2}|)\}$. Finally, put $\mathcal{B}:=\bigcup\{\mathcal{B}(x): x\in X\}$. \\
\indent a. Show that $\mathcal{B}$ is a basis for a topology on $X$. \\
\indent b. Show that the topology generated by $\mathcal{B}$ on $X$ is first-countable and separable. \\
\indent c. Endow $X$ with the topology generated by $\mathcal{B}$, and show that the subspace topology on $\mathbb{R}_{+}\times \{0\}$ is discrete. Conclude that $X$ is not second-countable.
\end{problem}

\textbf{Proof:} a. It's clear that $\bigcup\mathcal{B}=X$ by definition. Now consider $B(x,\epsilon_{1})\cap B(x',\epsilon_{2})$, where $B(x,\epsilon_{1})\in\mathcal{B}(x)$ and $B(x',\epsilon_{2})\in\mathcal{B}(x')$. If $x_{2}>0$ and $x'_{2}>0$, then the intersection is open in Euclidean sense, for $x''\in B(x,\epsilon_{1})\cap B(x',\epsilon_{2})$, there exists $B(x'',\epsilon)\subseteq\mathcal{B}(x)\cap\mathcal{B}(x')$, where we let $\epsilon<x_{2}''$, then $B(x'',\epsilon)\in\mathcal{B}$. Now suppose that $x_{2}=0$ and $x'_{2}>0$, and notice that $B(x,\epsilon_{1})\cap B(x',\epsilon_{2})$ never touches the $x$-axis, then by the same argument as in the first case, for $x''\in B(x,\epsilon_{1})\cap B(x',\epsilon_{2})$, there exists $B(x'',\epsilon)\subseteq\mathcal{B}(x)\cap\mathcal{B}(x')$, and $B(x'',\epsilon)\in\mathcal{B}$. Finally, if $x_{2}=0=x'_{2}$, then also discover that $B(x,\epsilon_{1})\cap B(x',\epsilon_{2})$ never touches $x$-axis, thus $\mathcal{B}$ is a basis for a topology on $X$. \\
\indent b. First show that the topology is first-countable. Suppose that $x=(x_{1},x_{2})$, where $x_{2}=0$. For any $O\in\mathcal{O}_{X}(x)$, $O$ must contain $\{x\}\cup B((x_{1},\epsilon),\epsilon)$, then there exists $1/m$, $m\in\mathbb{N}$ such that $x\in\{x\}\cup B((x_{1}, 1/m),1/m)\subseteq O$, there is a countable local basis for such $x$. Now for the case where $x_{2}>0$, the countable local basis is $B(x,\min\{x_{2},1/m\})$, because for any $O\in\mathcal{O}_{X}(x)$, deleting possible points on the $x$-axis, it is trivial that such local basis is valid. Finally, to prove separability, consider point set $S=\mathbb{Q}\times\mathbb{Q}_{\geq 0}$. For $x=(x_{1}, x_{2})$ where $x_{2}>0$, by discussion in the normal Euclidean space, for any open set $O$ containing $x$, $O\cap A\neq\emptyset$. When $x_{2}=0$, any open set containing $x$ must contain $\{x\}\cup B(x_{1}, \epsilon)$ for $\epsilon>0$, then it must intersect points in $S$, hence we find a countable dense set in $X$, Moore Plane is separable. \\
\indent c. For any subset $O\subseteq \mathbb{R}_{+}\times\{0\}$, $O=\left (\bigcup_{x\in\mathbb{R}_{+}\times\{0\}}\{x\}\cup B((x,\epsilon),\epsilon)\right )\cap \mathbb{R}_{+}\times \{0\}$ for any $\epsilon>0$, then every subset in $\mathbb{R}_{+}\times \{0\}$ is open, thus is discrete. Since $\mathbb{R}\times\{0\}$ is uncountable, it is not second-countable, then the ground space $X$ cannot be second-countable. \qed
\\
\begin{problem}
Let $\mathcal{F}$ stand for the set of all $\{0,1\}$-valued functions $f$ on $\mathbb{R}$ such that $f^{-1}(0)$ is a finite set. Let $\textbf{0}$ denote the self-map on $\mathbb{R}$ that equals to zero everywhere. \\
\indent a. Show that $\textbf{0}$ is not in $\text{cl}(\mathcal{F})$, if we view $\mathcal{F}$ as residing in $\textbf{B}(\mathbb{R})$. \\
\indent b. Show that $\textbf{0}\in\text{cl}(\mathcal{F})$, if we view $\mathcal{F}$ as residing in $\mathbb{R}^{\mathbb{R}}$ with the topology on $\mathbb{R}^{\mathbb{R}}$ being the topology of pointwise convergence.
\end{problem}

\textbf{Proof:} a. Consider the set $O:=\{f\in\textbf{B}(\mathbb{R}):\|f-\textbf{0}\|_{\infty}<2\}$, then $O$ is open in $\textbf{B}(\mathbb{R})$. Also, $O$ never intersects $\mathcal{F}$, since if so, $f^{-1}(1)$ is a finite set, and $f^{-1}(0)$ will be a infinite set for $f\in\mathcal{F}$, which contradicts the definition of $\mathcal{F}$. \\
\indent b. Write any basis element as $U(x_{1}, O_{1})\cap U(x_{2}, O_{2})\cap\dots\cap U(x_{n})$, where $U(x,O):= \{f\in\mathbb{R}^{\mathbb{R}}: f(x)\in O\}$ for $O\in\mathcal{O}_{\mathbb{R}}$, by the definition of topology of pointwise convergence. Then for any open set $\mathcal{U}\ni \textbf{0}$, $\mathcal{U}=\bigcup_{i\in I}U(x_{1,i}, O_{1, i})\cap\cdots\cap U(x_{n_{i}, i}, O_{n_{i}, i})$. Notice that since $\textbf{0}\in\mathcal{U}$, there must exists a $i\in I$ such that $0\in O_{j,i}$ for $j=1, \dots, n_{i}$, then choose $f$ such that $f\in U(x_{1,i}, O_{1, i})\cap\cdots\cap U(x_{n_{i}, i}, O_{n_{i}, i})$ for that particular $i$, specifically $f(x_{j, i})=0$ for $j=1, \dots, n_{i}$, then let $f(x)=1$ otherwise, then $f\in\mathcal{F}$, hence $\textbf{0}\in\text{cl}(\mathcal{F})$. \qed
\\
\begin{problem}
Let $I$ be a nonempty set, $X_{i}$ a topological space for each $i\in I$, and $X$ the cartesian product of $\{X_{i}: i\in I\}$. Then, it is easily verified that $\{\prod_{i\in I} O_{i}: i\in I\}$ is a basis for a topology on $X$. The topology that is generated by this basis is Hausdorff, and it is called the box topology on $X$. \\
\indent a. Explain why the box topology is finer than the product topology on $X$. Are the two topologies the same when $I$ is finite? \\
\indent b. Show that the map $f: \mathbb{R}\to\mathbb{R}^{\infty}$, defined by $f(x):= (x, x, \dots)$, is continuous when we view $\mathbb{R}^{\infty}$ as a topological space relative to the product topology, but not relative to the box topology. \\
\indent c. Show that the sequence $(x_{m})$ in $\mathbb{R}^{\infty}$, where $x_{m}:= (0, \dots, 0, m, m, \dots)$ with the first nonzero term being the $m$-th one, converges relative to the product topology, but not relative to the box topology.
\end{problem}

\textbf{Proof:} a. For product topology, only for finite index do we have $O_{i}$ rather than the whole space $X_{i}$, so the box topology has more open sets than product topology, thus is finer. If $I$ is finite, then the two are the same. \\
\indent b. First show that $f$ is not continuous in box topology, consider the preimage of $O:=(-1, 1)\times(-\frac{1}{2}, \frac{1}{2})\times\cdots$, then $f^{-1}(O)=\{0\}$, since if $f(x)=x\in O$ for $x\neq 0$, then there exists $1/m$ such that $x\notin (-1/m,1/m)$, hence $f$ cannot be continuous because $\{x\}$ is not open in $\mathbb{R}$. Secondly, for product topology, for any open set $O$, we may assume without loss of generality that it is of the form $O_{1}\times O_{2}\times\cdots\times O_{m}\times \mathbb{R}\times\cdots$, then $f^{-1}(O)=\bigcap_{i=1}^{m}O_{i}$, thus is open in $\mathbb{R}$, conclude that $f$ is not continuous in box topology but continuous in product topology. \\
\indent c. First prove for the product space. Claim that the sequence converges to $\textbf{0}=(0, 0, \dots)$, then for any open set $O$ containing $\textbf{0}$, write $O=\prod_{i=1}^{\infty}O_{i}$, and for all but finitely many $i$, $O_{i}=\mathbb{R}$, each $O_{i}$ contains $0$. Suppose that $k\in\mathbb{N}$ is the largest index where $O_{i}\neq \mathbb{R}$, then for $x_{n}=(0, 0, \dots, n, n\dots)$, $n\geq k$, obviously $x_{n}\in O$, hence $(x_{m})$ converges in product topology. For box topology, suppose that $(x_{m})$ converges to $x$, then consider the open set in box topology $O:= (-2x, 2x)\times (-2x, 2x)\times\cdots$, for $m>2x$, $x_{m}\notin O$, thus not convergent. \qed
\\
\begin{problem}
Prove: For any positive integer $n$, prove that $O(n)\simeq SO(n)\times \{-1, 1\}$.
\end{problem}

\textbf{Proof:} Notice that for $A\in O(n)$, $AA^{T}=I$, which implies that $\det(A)=\pm 1$. Consider the function $f: O(n)\to SO(n)\times \{-1, 1\}$ by $M\mapsto (T(M), \det(M))$, where
$$
T(M)= \begin{pmatrix} \det(M) & 0 &\cdots & 0 \\ 0 & 1 & \dots & 0\\ 0 & 0 & \ddots &\\ 0 & 0 & \cdots & 1\end{pmatrix} \circ M
$$
then clearly $T$ is bijective, since for any $(O, i)\in SO(n)\times \{-1, 1\}$, there exists $M\in O(n)$ such that $M\in O(n)$, where all entries are the same except possibly for the first row where all entries are negative of $O$, injectivity is given by matrix multiplication. Also, continuity of $f$ and its inverse is clear from continuity of linear transformation and determinant, hence $O(n)\simeq SO(n)\times \{1, -1\}$.\qed
\\
\begin{problem}
Let $X$ and $Y$ be topological spaces, and take any $f: X\to Y$. Consider the map $F: X\to \text{graph}(f)$, defined by $F(x):= (x, f(x))$, and show that $F$ is a homeomorphism iff $f\in\textbf{C}(X,Y)$.
\end{problem}

\textbf{Proof:} Suppose that $F$ is a homeomorphism, then $F(x)=(x,f(x))$ is continuous, then $f(x)\in\textbf{C}(X,Y)$ as a component map. Conversely, if $f\in\textbf{C}(X,Y)$, then $F(x)=(x,f(x))$ is continuous. Note that $F$ is bijective, since if $x_1\neq x_2$, then $(x_1,f(x_1))\neq (x_2,f(x_2))$ for sure, and surjectivity is given by definition. For any open set $O=O_{X}\times O_{Y}\subseteq \text{graph}(f)$ for $O_{X}\in\mathcal{O}_{X}, O_{Y}\in\mathcal{O}_{Y}$, $F^{-1}(O)=O_{X}\cap f^{-1}(O_{Y})$ is open , then $F^{-1}$ is continuous, thus is a homeomorphism. \qed

\end{document}